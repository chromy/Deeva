\documentclass[11pt]{article}

\usepackage{listings}
\usepackage{graphicx}

\renewcommand{\topfraction}{0.9}    % max fraction of floats at top
\renewcommand{\bottomfraction}{0.8}

\begin{document}

\title{Planing Report}
\author{Kritaphat Sonsri-in, Xueqi Chen, Hector Dearman, Alina Draganescu, Felix De Souza}

\maketitle

\section{Introduction}

Our group was assigned to the Project Java Debugger which will be designed for first year computing students. After we discussed with Tristan, who proposed this project, we understood that the problem while teaching Java to first year students is that students tend to get really dependant on IDE like Eclipse with functionalities like auto-fill etc. Tristan has tried alternatives like using JDB however it sometimes gets out of sync and breaks. This is why he wants a simpler version of IDE that is more light-weighted. 

\section{Additional Notes}

Was this.

\section{Agile Programming}
We decided to use Agile Programming as we go along. A Trello board was then created for a better project management process. There are several different swimlanes on the board: ‘Proposed’, which allows all members to throw in ideas; ‘In analysis’, where people talk about their ideas and have a discussion with group members to see if the story is feasible. If it is, the card will be then moved to ‘In estimation’ where we gather together to use planning pokers to give a complexity points to the story. Otherwise, the story will be moved to ‘Discarded’. The reason why we want to keep track of what we have discarded is that we might have thought we would not have time for it but in reality we did have time at the end, we might want to move it back and reconsider it. One of our group members has hand-made a set of pokers just for estimation. It contains Fibonacci number 0 - 13 and ‘infinity’ if we think it’s impossible to do as well as ‘coffee time’ if we think we need a break. After estimation, the card will be assigned to group members with a set due date and be moved into ‘In development’. Once the development is complete, the card should be moved into ‘In review’. This is when we need someone to review our code to improve our styling and pattern-usage. We have been using the board to manage some of the tasks and found it really efficient since we are now given with specific tasks and deadlines comparing with project managing in first and second year.

\includegraphics[width=\textwidth]{TrelloBoard.png}

\end{document}
