\documentclass[11pt, a4paper]{article}

\usepackage{listings}
\usepackage{subfigure}
\usepackage{graphicx}
\usepackage{titling}
% \usepackage[margin=1.8cm, includefoot]{geometry}
\usepackage{parskip}
\setlength{\parindent}{0cm}
\usepackage{titlesec}

% Select what to do with todonotes: 
% \usepackage[disable]{todonotes} % notes not showed
\usepackage[draft]{todonotes}   % notes showed

\newlength\longest

\titlespacing\section{0pt}{6pt plus 2pt minus 2pt}{0pt plus 0pt minus 2pt}
\titlespacing\subsection{0pt}{4pt plus 2pt minus 2pt}{0pt plus 0pt minus 2pt}
\newcommand{\subtitle}[1]{
  \posttitle{
    \par\end{center}
    \begin{center}\large#1\end{center}
    \vskip0.5em}
}

\setlength{\droptitle}{-4em}

\newcommand{\cmd}[1]{{\tt #1}}

\begin{document}

\title{Deeva}
\subtitle{Final Report}
\author{Kritaphat Sonsri-in, Xueqi Chen, Hector Dearman, \\Alina Draganescu, Felix de Souza}

\maketitle

% Report editing rules:
% Each sentence on a new line.
% Only edit one section before committing.
% Push after every commit.
% ONLY EDIT ONE SECTION BEFORE COMMITTING.



% Three main goals:
% 1. Fill a tool hole for first year Java course
% 2. Foster a strong mental model of imperative programming
% 3. Introduction to debugging

% Introduction 
%   Set the scene (‘motivation’) 
%   State the problem you are trying to solve (‘objective(s)’) 
%   Summarise what you achieved (‘contributions’) 
% Design & implementation 
%   Detail your design (why did you do it this way?) 
%   Summarise key implementation details (how did you do it? what tools did you use?) 
% Evaluation 
%   Summarise testing procedures (+ relevant testing results) 
%   Evaluate your deliverables, e.g. in terms of performance, usability, usefulness… 
%   (how successful was the project?) 
% Conclusion and future extensions 
%   Say what you’ve concluded from doing the work and how you’d build on it 
% Project management 
%   Planning, group organisation, breakdown + task allocation etc


\section{Executive Summary}

% One line summary:
Deeva is a simple Java debugger for teaching which allows graphical introspection of a running program.

% What is a debugger?
``Bugs'' are a type of software defect typically introduced though a mismatch between the programmers mental model of the system and the system itself.
A ``debugger'' then is a program that allows programmers to inspect the state and flow of a running system so they can understand the mismatch and hence fix the defect.

% Why have we built one?
In the first year introductory Java course many students are encountering imperative programming for the first time so there is a strong desire \todo{``by course leaders"?} to minimise `magic' particularly in the form of complex IDEs.

Specifically students are encouraged to use a text editor (like \cmd{vim} or \cmd{emacs}) and invoke the Java compiler (\cmd{javac}) and virtual machine (\cmd{java}) directly in order develop a deeper and more transferable understanding of programming.
Unfortunately this attempt is hamstrung by the absence of a good Java debugger decoupled from an IDE.
Without a strategy to fix broken programs students quickly discover bad habits ranging from the awful, shotgun debugging, to the merely poor, print statements.

We built Deeva to chiefly to fill this hole in the Java ecosystem and Lab infrastructure but it was designed with two additional goals in mind:
first to use Deeva to help foster a strong mental model of imperative programming and to introduce students to powerful introspection of software systems.

% Summary of report content:
In this report we will first make the case for the necessity of a tool like Deeva then enumerate our goals and evaluate how well Deeva meets these goals before moving on to discuss the design, implementation and management of the project. 
Lastly we conclude with our final evaluation of the project and directions for future work.

\section{Motivation}
Despite the existence of powerful tools to introspect running programs many programmers continue to debug programs using print statements.
Since 
Over half of first year computing students at Imperial have no previous experience of imperative programing
A strong mental model of programming basics are highly predictive of eventual achievement.

\section{Objectives}
\section{Contributions}
\section{Design}
\section{Implementation}
\section{Project Management}

To ensure effective development of the product, weekly supervisor meetings were scheduled.
In these meetings, quality assurance was the main aim, making sure we were on the right track with our development, delivering correct features. 
This avoided unnecessary misunderstandings and extra time spent on creating unneeded features for the product. 

Further to the supervisor meetings, weekly group meetings ensured that all group members are in sync about the progress and direction of the project. 
During these meetings, the information gathered from the supervisor meetings would be recapped and the targets for the upcoming week would be set. 
Thus, tasks could be divided among group members, with everyone knowing what their colleagues are working on in this week, making group and individual efforts significantly more effective.
\begin{figure}[h!]
\centering
\includegraphics[height=80mm,width=130mm]{estimation.jpg}
\end{figure}

To aid the management of the project on a holistic and timely perspective, several tools were used, which will be outlined in the following section.

\subsection{Tools}
\begin{description}

\item[Trello] \hfill \\
\begin{figure}[h!]
\centering
\includegraphics[height=80mm,width=130mm]{Trello.png}
\end{figure}

We used Trello as our project management tool and as an Information Radiator\footnote{\tt{https://www.atlassian.com/wallboards/information-radiators.jsp}}.
It also allows us to track our progress and average velocity. 
We decided to use Trello as it fulfills most of the key features required for such a tool: it is easy to use and update, relatively flexible and allows for communications between members without too much interruption. 
We decided to use an electronic version as they are a more manageable alternative to physical boards. 
We also found it extremely useful when it came to keeping everyone informed on the general progress and keeping all the information in one compact environment.

A Trello board was created for a better project management process. 
There are several different swimlanes on the board: ``Proposed'', which allows all members to throw in ideas; ``In Analysis'', where people talk about their ideas and have a discussion with group members to see if the story is feasible. 
If it is, the card will be then moved to ``In Estimation'' where we would use Planning Poker to assign the story with a complexity points.
Otherwise, the story will be moved to ``Discarded''.
The reason why we want to keep track of what we have discarded is that we might have thought we would not have time for it, but in reality we may have time at the end.
We might want to move it back and reconsider it. 
After ``In Estimation'', the card will be assigned to group members with a set due date and be moved into ``In Development'' on the Trello board. 
Once the development is complete, the card should be moved into ``In Review''. 
This is when we meet as a group and review our code to improve our style and pattern-usage.

\item[Planing Poker] \hfill \\
\begin{figure}[h!]
\centering
\includegraphics[height=80mm,width=130mm]{planningPokers.jpg}
\end{figure}

We are using Planning Poker in order to estimate stories in each cycle of development. 
This method is known to avoid anchoring and will produce more accurate, less optimistic story point estimations\footnote{\tt{http://en.wikipedia.org/wiki/Planning\_poker\#Planning\_poker\_benefits}}. This technique takes some time to get used to because initially the story point bared little meaning to us.
As we estimated more tasks, the story points became more meaningful hence the poker game proved to be a very efficient method.

For example, it helped determine when people aren't really talking about the same scope for a certain task.
Frequently we have a task like ``Set up the back end." and one person gives a task a 2 while another gives it an 8.
The first person thinks the task title means ``Write some stubs so the middleware can integrate." while the second thinks it means ``Implement the backend up to the minimal specification.". 
So using planing poker has already helped us make sure we're all on the same page.
One of our group members hand-made a set of poker cards just for estimation.
It contains the Fibonacci numbers up to 13 including 0 and infinity - if the task was unfeasible in the time given, as well as ``coffee time'' if we think we needed a break. 

\item[Facebook group] \hfill \\
Creating a Facebook group proved very efficient for communicating meeting times and general enquiries that were too conversational for Trello.  

\item[Google Docs] \hfill \\
We used Google Docs for collaborative editing on reports, plans and meeting summaries in real time.
  
\item[Github and Travis CI] \hfill \\
Github is a hosted source control which somebody else manages and is easily accessible from anywhere and also plays nicely with the hosted Continuous Integration tool - Travis CI\footnote{\tt{http://travis-ci.com/}}.
\end{description}
\section{Extensions}
\section{Conclusion}

\clearpage
\thispagestyle{empty}
\null\vfill
\begin{center}
\settowidth\longest{``A process cannot be understood by stopping it."}
\parbox{\longest}{%
  \raggedright{%
  ``A process cannot be understood by stopping it." \\
  }   
  \raggedright{\emph{First Law of the Mentat -- Dune}}\par%
}
\end{center}
\vfill\vfill
\clearpage

\end{document}
