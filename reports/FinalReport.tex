\documentclass[11pt, a4paper]{article}

\usepackage{listings}
\usepackage{subfigure}
\usepackage{graphicx}
\usepackage{titling}
% \usepackage[margin=1.8cm, includefoot]{geometry}
\usepackage{parskip}
\setlength{\parindent}{0cm}
\usepackage{titlesec}

% Select what to do with todonotes: 
% \usepackage[disable]{todonotes} % notes not showed
\usepackage[draft]{todonotes}   % notes showed

\newlength\longest

\titlespacing\section{0pt}{6pt plus 2pt minus 2pt}{0pt plus 0pt minus 2pt}
\titlespacing\subsection{0pt}{4pt plus 2pt minus 2pt}{0pt plus 0pt minus 2pt}
\newcommand{\subtitle}[1]{
  \posttitle{
    \par\end{center}
    \begin{center}\large#1\end{center}
    \vskip0.5em}
}

\setlength{\droptitle}{-4em}

\newcommand{\cmd}[1]{{\tt #1}}

\begin{document}

\title{Deeva}
\subtitle{Final Report}
\author{Kritaphat Sonsri-in, Xueqi Chen, Hector Dearman, \\Alina Draganescu, Felix de Souza}

\maketitle

% Report editing rules:
% Each sentence on a new line.
% Only edit one section before committing.
% Push after every commit.
% ONLY EDIT ONE SECTION BEFORE COMMITTING.



% Three main goals:
% 1. Fill a tool hole for first year Java course
% 2. Foster a strong mental model of imperative programming
% 3. Introduction to debugging

% Introduction 
%   Set the scene (‘motivation’) 
%   State the problem you are trying to solve (‘objective(s)’) 
%   Summarise what you achieved (‘contributions’) 
% Design & implementation 
%   Detail your design (why did you do it this way?) 
%   Summarise key implementation details (how did you do it? what tools did you use?) 
% Evaluation 
%   Summarise testing procedures (+ relevant testing results) 
%   Evaluate your deliverables, e.g. in terms of performance, usability, usefulness… 
%   (how successful was the project?) 
% Conclusion and future extensions 
%   Say what you’ve concluded from doing the work and how you’d build on it 
% Project management 
%   Planning, group organisation, breakdown + task allocation etc


\section{Executive Summary}

% One line summary:
Deeva is a simple Java debugger for teaching which allows graphical introspection of a running program.

% What is a debugger?
``Bugs'' are a type of software defect typically introduced though a mismatch between the programmers mental model of the system and the system itself.
A ``debugger'' then is a program that allows programmers to inspect the state and flow of a running system so they can understand the mismatch and hence fix the defect.

% Why have we built one?
In the first year introductory Java course many students are encountering imperative programming for the first time so there is a strong desire \todo{``by course leaders"?} to minimise `magic' particularly in the form of complex IDEs.

Specifically students are encouraged to use a text editor (like \cmd{vim} or \cmd{emacs}) and invoke the Java compiler (\cmd{javac}) and virtual machine (\cmd{java}) directly in order develop a deeper and more transferable understanding of programming.
Unfortunately this attempt is hamstrung by the absence of a good Java debugger decoupled from an IDE.
Without a strategy to fix broken programs students quickly discover bad habits ranging from the awful, shotgun debugging, to the merely poor, print statements.

We built Deeva to chiefly to fill this hole in the Java ecosystem and Lab infrastructure but it was designed with two additional goals in mind:
first to use Deeva to help foster a strong mental model of imperative programming and to introduce students to powerful introspection of software systems.

% Summary of report content:
In this report we will first make the case for the necessity of a tool like Deeva then enumerate our goals and evaluate how well Deeva meets these goals before moving on to discuss the design, implementation and management of the project. 
Lastly we conclude with our final evaluation of the project and directions for future work.

\section{Motivation}
Despite the existence of powerful tools to introspect running programs many programmers continue to debug programs using print statements.
Since 
Over half of first year computing students at Imperial have no previous experience of imperative programing
A strong mental model of programming basics are highly predictive of eventual achievement.

\section{Objectives}
\section{Contributions}
\section{Design}
\section{Implementation}
\section{Project Management}
\section{Extensions}
\section{Conclusion}

\clearpage
\thispagestyle{empty}
\null\vfill
\begin{center}
\settowidth\longest{``A process cannot be understood by stopping it."}
\parbox{\longest}{%
  \raggedright{%
  ``A process cannot be understood by stopping it." \\
  }   
  \raggedright{\emph{First Law of the Mentat -- Dune}}\par%
}
\end{center}
\vfill\vfill
\clearpage

\end{document}
